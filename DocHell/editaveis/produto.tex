\chapter{Produto}\label{produto}

\section{Introdução a Ferramenta Lettuce}

    Neste tópico, será descrito o guia criado pelo grupo quanto ao uso da
    ferramenta \textit{lettuce} para automatização de cenários de testes
    baseados em BDD. Vale ressaltar que o guia aqui apresentado visa a
    integração dessa ferramenta com o framework para desenvolvimento web django.
    Apesar do django ser implementado usando a linguagem python, o grupo não
    pode garantir que o tutorial aqui descrito poderá ser aplicado de forma
    literal para projetos em python que não estejam usando o django. Por fim,
    vale ressaltar que o grupo ainda planeja criar vídeos mostrando os passos
    descritos neste tutorial, facilitando assim a visualização dos mesmos.

\section{Instalação de Dependências e Configurações}

    Neste tópico será mostrado como instalar e configurar a ferramenta lettuce
    para ser usada juntamente com projetos django. Vale ressaltar que o
    tutorial parte do pressuposto de que o framework django e o gerenciador de
    pacotes para python, pip, já estejam instalados na máquina do usuário.

    Para instalar o lettuce ao projeto, basta usar o comando:

    \begin{verbatim}
    $pip install lettuce
    \end{verbatim}
    
    Uma vez instalada a ferramenta, é necessário adicionar a mesma o lettuce à
    variável INSTALLED\_APPS do django. Tal variável é na verdade uma lista
    contendo os diferente apps que compõem o projeto. Desa forma, basta
    adicionar a lista de variáveis a instrução "lettuce.django" confome o código
    a seguir:

    \begin{verbatim}
    INSTALLED_APPS = (
        'django.contrib.auth',
        'django.contrib.admin',

        # other apps

        'lettuce.django',
        )
    \end{verbatim}

    Uma vez configurada as variáveis de ambiente do django, os testes já podem
    começar a ser escritos. Entretanto, é necessário explicar antes como
    funciona a estrutura de testes para um projeto lettuce. A ferramenta
    necessita que os testes relacionados a ela estejam dentro de uma pasta
    chamada "features". Dentro desta pasta, dois tipos de arquivos devem estar
    presentes:


    \begin{itemize}
         \item \textbf{feature}: Arquivo usado para descrever o cenário de
             teste. Tal arquivo é basicamente a definição do passo-a-passo que
             será seguido pela ferramenta na hora de executar o teste. A
             estrutura desse arquivo pode ser vista no código á seguir:


	     \begin{verbatim}
	     Feature: Register user
   	     As an admin
         I want to register an user
	     So I can have a new user on the system

             Scenario: Register user
             Given I am on main page
             And I enter the user name "Test"
             When I press the button register
             Then I should see "Test registered with success"
	     \end{verbatim}	

	     Esse arquivo deve possuir duas estruturas básicas, a feature, que é seria a descrição
	     da história de usuário a cujo teste esta relacionado. Além da própria
	     história de usuário, é necessário também a descrição do cenário que
	     vai ser testado para a história descrita. A ferramenta lettuce usa a
	     estrutura gherkin para escrever seus cenários de teste, facilitando
	     assim que a mesma possa ser escrita por qualquer membro do projeto, não
	     necessitando necessariamente de um conhecimento técnico.

         Por fim, vale dizer que este arquivo precisar ter a extensão
         ".feature". Além disso, é possível configurar este arquivo para
         facilitar ainda mais a escrita dos testes relacionados ao cenário
         descrito no arquivo, entretanto, isso será melhor explorado na próxima
         seção.

        \item \textbf{steps}: Arquivo python que irá conter a estrutura
            necessária para a execução do cenário descrito no arquivo .feature
            relacionado. Para que este arquivo seja executado com sucesso, é
            necessário que o arquivo .feature esteja presente no mesmo diretório
            que ele. A estrutura de teste desse arquivo será explicada na
            próxima seção do guia.

     \end{itemize}

    

\section{Estrutura dos Testes}

    Neste tópico, a estrutura básica de um arquivo de teste da ferramenta
    lettuce será descrita. Todo o teste deve ser escrito dentro do arquivo
    steps.py. Este será o script python responsável pela execução dos testes
    propriamente ditos.

    Antes de mostrar como escrever os testes baseados em um arquivo .feature, é
    necessário antes importar algumas bibliotecas. Dessa forma, dentro do
    arquivo steps.py, adicione as seguintes linhas:

    \begin{verbatim}
    from lettuce import *
    \end{verbatim}

    Este comando irá garantir que os módulos associados ao lettuce serão
    importados. Com esse passo realizado, é necesário fazer ainda sim outra
    configuração. Considerando que o lettuce é uma ferramenta black box em sua
    essência, os testes implementados pela ferramenta serão todos funcionais.
    Entretanto, como as aplicações escritas em django são aplicações web, para
    realizar o teste funcional, é necessário um browser para que a aplicações
    possa ser de fato acessada. Sendo assim, é necessário isolar o browser de
    alguma forma dos testes, para que o mesmo não se torne uma dependência.
    Felizmente, a framework django possui uma classe específica para simular o
    uso de um browser, a classe Client. Para importá-la no projeto, é necessário
    adicionar a seguinte linha no começo do steps.py:

    \begin{verbatim}
    from django.test import Client
    \end{verbatim}

    Com este módulo importado, é necessário agora instanciar um objeto Client.
    Vale ressaltar que o mesmo será usado ao longo de todos os testes, sendo
    assim, tal objeto deve poder ser usado por qualquer caso de teste e deve ser
    instanciado antes dos casos de teste serem executados. Para isso, pode-se
    usar o seguinte comando:

    \begin{verbatim}
    @before.all
    def set_browser():
        world.browser = Client()
    \end{verbatim}

    A anotação \textit{@before.all} comunica ao script de teste que o método
    deve ser executado antes da execução dos testes. Dentro do método
    set\_browser, uma instância de client está sendo criada e associado a uma
    variável global. A declaração \textbf{world} antes do nome de uma variável
    garante que a mesma será global, ou seja, será visível para qualquer caso de
    teste.

    Com essas configurações, pode-se enfim começar a escrita de um caso de teste
    propriamente dito. Todo caso de teste deve possuir um passo associado a ele,
    ou seja, cada caso de teste deve estar associado a um passo definido na
    estrutura \textit{scenario} presente no arquivo .feature associado. Para
    isso, todo caso de teste deve possuir a seguinte anotação:

    \begin{verbatim}
    @step{"Given I am in the home page"}
    \end{verbatim}

    Vale ressaltar que dentro da anotação \textit{step}, deve-se conter a mesma
    definição de passo criada em um arquivo .feature. Por exemplo, se um dos
    passos descritos no arquivo .feature é "When I click on button search", o
    teste relacionado a este passo deve possuir uma anotação \textit{step}
    exatamente igual ao descrito na feature.

    Com a anotação \textit{step} definida, pode-se então descrever a estrutura
    do método de teste propriamente dito. Logo após a anotação, deve-se declarar
    um método que possui sempre um parâmetro default, chamado de \textit{step}.
    Um exemplo de declaração de um caso de teste para a ferramenta lettuce pode
    ser visto logo a seguir:

    \begin{verbatim}
    @step('I am on application home page"')
    def i_am_on_application_home_page(step):
        #do something here
    \end{verbatim}

    O caso de teste deve então possuir código relacionado ao \textit{step}
    especificado. Neste cenário, pode-se usar desde outro frameworks de teste
    para python, como o nose, para permitir o uso de asserções. Entretanto,
    o step será considerado como válido apenas se as instruções dentro do
    método de teste não obtiverem problema na hora da execução, ou seja, não é
    ncessário usar asserções para que o teste passe.

    Uma feature interessante, que precisa ser descrita, é a possibilidade de um
    passo descrito dentro do arquivo .feature ser capaz de passar um parâmetro
    para um método de teste. Para isso, é necessário que a variável dentro de um
    passo no arquivo ,feature seja especificada entre aspas. Um exemplo desse
    modelo pode ser visto a seguir:

    \begin{verbatim}
    And I type "205" in the field search lines
    \end{verbatim}

    Observa-se que o 205 está entre aspas, sendo assim, o mesmo pode ser
    considerado uma variável. Com este padrão definido, pode-se assim capturar
    tal variável de forma mais fácil dentro da anotação \textit{step} antes do
    método de teste. Um exemplo disso pode ser observado a seguir:

    \begin{verbatim}
    @step(And I type "(.*)" in the field search lines)
    def and_i_type_value_in_the_field_seach_lines(steo, value):
        #do something here...
    \end{verbatim}

    Pode-se ver que dentro da anotação, existe agora uma expressão regular ou
    regex dentro dela, isso é feito para capturar o padrão da variável. Qualquer
    regex pode ser usada neste cenário, facilitando assim capturar determinados
    padrões de variáveis. Vale ressaltar também que o próprio método de teste
    deve também receber a variável ou variáveis capturadas pela regex quando se
    usa tal funcionaliddade.

    Por fim, uma vez que já tenha escrito algum teste ou completado todos os
    passos para o cenário descrito no arquivo .feature, pode-se rodar os testes
    com o comando mostrado a seguir:

    \begin{verbatim}
    $python manage.py harvest
    \end{verbatim}

    Tal comando irá percorrer todos os apps do projeto procurando por uma pasta
    feature\. Uma vez encontrada, o mesmo irá então executar o arquivos steps.py
    presente no diretório. Sendo assim, a pasta feature crida tem que estar
    dentro do diretório raiz do app relacionado para que este comando funcione
    de forma adequada. Uma vez executado, o mesmo irá mostrar quais passos foram
    executados, especificando quais passaram e quais falharam, A visualização
    dos resultados será melhor explicada na próxima seção.

\section{Exemplo de Funcionamento}

    Baseado no guia proposto, foi desenvolvido um cenário de teste completo
    usando a ferramente Lettuce para o software Busine.me. Para a criação de um
    caso de teste, foi necessário escrever uma das funcionalidades do aplicativo
    como uma história de usuário além de também escrever um dos seus critérios
    de aceitação.  A funcionalidade escolhida foi a busca por linhas de ônibus por
    meio da aplicação, e o cenário de aceitação escolhido foi para o
    procedimento de busca para uma linha válida. Tanto a história de usuário e
    seu critério de aceitação podem ser vistos a seguir:

    \begin{verbatim}
    Feature: Search busline
    In order to check a busline
    As an user
    I want to search for a busline

    Scenario: Search by line number
        Given I am on Busine.me home page
        And I type "205" in the field search lines
        When I press "Search"
        Then I should see "Resultados"
        And I should see "2" buslines that contains "205" in their numbers
   \end{verbatim}

   Vale ressaltar que a história de usuário e o cenário descritos estão em
   inglês para seguir o padrão de codificação usado pela própria ferramenta.
   Dessa forma, qualquer linguagem pode ser usada, contanto que a convenção da
   linguagem gherkin seja usada, ou seja, as anotações e conectivos, como
   \textit{And} e \textit{Given} devem estar em inglês.

   Com o arquivo .feature criado, o arquivo steps.py foi também criado. A
   estrutura do arquivo pode ser vista na figura a seguir:

    \begin{figure}[h!]
        \centering
        \includegraphics[width=0.9\textwidth]{figuras/steps.eps}
        \caption{Código desenvolvido para verificar o critério de aceitação
        definido}
    \end{figure}

    As marcações nesta figura servem para explicar os seguinte conceitos:

    \begin{itemize}
        \item \textbf{(1)}: Aqui estão as bibliotecas necessárioas para a
            execução dos testes. Dentre elas, pode ser visto a biblioteca
            \emph{django.test.client} que é usada para usar o browser de teste
            citada na parte de configuração do guia. Outra biblioteca que vale a
            pena ser comentada é a \emph{nose.tools} que é uma biblioteca python
            usada para criação de testes unitários. Dessa forma, a mesma suporta
            métodos associados a asserção, como o \textit{assert\_equals} que é
            usado no arquivo.
        
        \item \textbf{(2)}: Aqui está declarada a função que realiza a
            configuração usada antes de qualquer teste ser iniciado. Esta função
            está basicamente instanciando um browser de teste, \textit{Client} e
            especificando as outras variáveis necessárias.

        \item \textbf{(3)}: Caso de teste para o primeiro passo do cenário de
            aceitação especificado. Basicamente, uma requisição está sendo usada
            para acessar a home page da aplicação Busine.me.

        \item \textbf{(4)}: Caso de teste usado para configurar para setar o
            valor a ser procurado pela aplicação.

        \item \textbf{(5)}: Caso de teste usado para verificar o terceiro passo
            do cenário de aceitação. Ele basicamente manda uma requisição do
            tipo \textit{POST} para a página de busca usando o valor
            especificado no caso de teste anterior. Após mandar a requisição, o
            seu código é verificado para saber se a requisição foi bem sucedida.

        \item \textbf{(6)}: Caso de teste usado para verificar se a página
            carregada é a pretendida ao verificar se a palavra "Resultados" está
            dentro do html recebido como resposta.

        \item \textbf{(7)}: Caso de teste usado para verificar se a página
            carregada contém o número de linhas especificado pelo último passo
            do cenário de aceitação. Basicamente, checa-se o html recebido e
            conta-se o número de vezes que o valor pesquisado aparece no corpo
            do html. 

    \end{itemize}

    Vale ressaltar que os valores setados no arquivo .feature são todos
    capturados no testes do steps.py por meio do uso de expressões regulares ou
    regex.

    Ao executar os testes, a seguinte imagem é exibida:

    \begin{figure}[h!]
        \centering
        \includegraphics[width=0.9\textwidth]{figuras/results_1.eps}
        \caption{Resultado da execução dos testes usando python manage.py 
        harvest}
    \end{figure}

    Pode ser ver que todos os cenários passaram. Entrentanto, o comando "python
    manage.py harvest" busca por features em todos os apps do projeto, dessa
    forma, como apenas foi criada features em um dos apps do projeto, a
    ferramenta indica que nenhuma feature foi encontrada nos outros apps do
    projeto.

    Apesar do teste atender às necessidades levantadas, o mesmo ainda pode
    apresentar algumas complicações. Quando se pensa em usar uma ferramenta como
    o lettuce, em tese, partes especificas do código deveriam ser evitadas na
    hora da contrução do steps.py. Entretanto, para o arquivo de teste feito,
    aspectos como o tipo de requisição feita e o nome da variável na requisição
    precisaram ser conhecidos, sendo assim, um nova forma de teste foi
    implementada. Esta nova forma de teste usa duas novas ferramentas:

    \begin{itemize}

        \item \textbf{splinter}: Ferramenta usada para testes de aceitação
            automatizados para python, ou seja, tal ferramenta permite
            abstração de detalhes de desenvolvimento para interação com a
            aplicação. Usando o splinter, pode-se pedir que o mesmo clique em
            algum botão na tela, ou digite algum valor em algum campo de texto,
            sendo necessário detalhes como o nome dos campos a serem
            interagidos. Entretanto, a ferramenta splinter precisa de um browser
            para ser usado, é a versão usada para testes do django não é
            suportada.

            Para instalar a ferramenta. basta executar o seguinte comando no
            terminal:

            \begin{verbatim}
            pip install splinter
            \end{verbatim}

        \item \textbf{zope.testbrowser}: O problema do browser do splinter foi
            resolvido usando essa ferramenta. Ao usar o splinter com um browser
            default, para todos os testes ele abre uma instância do browser e
            começa a executar os comando no mesmo. Por exemplo, caso o browser
            default seja o firefox, o splinter iria abrir tal browser durante os
            testes e executar os comandos nele. Para evitar esse problema
            durante a execução dos testes, um browser que rode em background foi
            usado, chamado de zope.browser, permitindo assim que o splinter
            execute os testes sem abrir nenhum browser gráfico.

            Para instalar tal ferramenta, basta executar o comando:

            \begin{verbatim}
            pip install splinter[zope.testbrowser]
            \end{verbatim}

        \end{itemize}


    Com essa nova abordagem de teste, um novo cenário de aceitação foi usado,
    que pode ser visto a seguir:

    \begin{verbatim}
    Feature: Search busline
    	In order to check a busline
   	As an user
    	I want to search for a busline

    Scenario: Search by line number
        Given I am on Busine.me home page
        And I type "205" in the field search lines
        When I press "Search"
        Then I should see "Resultados"
        And I should see "0.205 \n Rota:Gama Sul-Leste/"M" Norte ( Comercial) \n VIA ACES. HRG/OESTE - GAMA \n Percurso: 45.78 KM \n Tarifa: R$3.0"
    \end{verbatim}

    Basicamente, a mundaça feita foi apenas no última passo do cenário de
    aceitação, pois a validação com o splinter permite um acesso mais fácil a
    alguns campos do html recebido como resposta.

    O arquivo steps.py para o uso da ferramenta splinter pode ser visto na
    Figura 03:

     \begin{figure}[h!]
        \centering
        \includegraphics[width=0.9\textwidth]{figuras/splinter.eps}
        \caption{Arquivo steps.py usando a ferramenta splinter para auxiliar nos
        testes}
    \end{figure}

    Para este novo arquivo de teste, algumas considerações precisam ser feitas
    quanto a sua estrutura:

    \begin{itemize}

        \item \textbf{(1)}: Pode-se ver que uma nova biblioteca precisou ser
            importada, que seria a \textit{splinter.browser}, é por meio dela
            que uma isntância do browser para o teste deverá ser criada.

        \item \textbf{(2)}: Na função que executa antes de todos os testes, o
            browser é instanciado, lembrando que o browser é a ferramenta
            zope.browser.

        \item \textbf{(3)}: Caso de teste usado para acessar a página inicial da
            aplicação Busine.me. Tal caso de teste é igual ao definido no
            arquivo steps.py sem a ferramenta splinter.

        \item \textbf{(4)}: Caso de teste usado para prencher o campo de busca
            da aplicação.

        \item \textbf{(5)}: Caso de teste usado para mandar a requisição de
            busca. Aqui pode ser visto uma das facilidades e se usa o splinter.
            Basta usar comando como \textit{click} para enviar a requisição,
            não mais sendo necessário saber o tipo de requisição necessária.

        \item \textbf{(6)}: Caso de teste usado para verificar se a página de
            resultados foi carregada. Mais uma vez, pode-se ver uma das
            facilidades do splinter, pois o mesmo permite a interação com o html
            recebido de forma mais clara.

        \item \textbf{(7)}: Caso de teste usado para verificar se as informações
            necessárias estão presentes na página.

    \end{itemize}

    Assim como o steps.py anterior, este também coleta os valores passados pelo
    passos do cenário de aceitação por uso de expressões regulares. Ao executar
    tais casos de teste, obtêm-se o seguinte resultado:

      \begin{figure}[h!]
        \centering
        \includegraphics[width=0.9\textwidth]{figuras/results_2.eps}
        \caption{Resultado dos testes usando o splinter ao rodar o comando
        python manage.py harvest}
    \end{figure}

    Pode-se ver que os cenários também passaram todos, mostrando que ambas as
    aborsagens podem ser usadas, mas vale ressaltar que a segunda abordagem. que
    usa o splinter, gera mais dependênias no projeto, o que pode ser um
    problema.

