\chapter[Webserver]{Cliente Ping}

\section{Linguagem de programação}

    A linguagem de programação adotada foi o C++, e também foi usado a linguagem
    python para implementar o exemplo que já estava no código ping.py

\section{Sistema operacional}

    O sistema operacional usada para implementar a solução proposta foi a distro
    linux Debian.

\section{Ambiente de desenvolvimento}

    Para o desenvolvimento da aplicação  foram usados: um editor de texto e um
    compilador.

    O editor de texto escolhido foi o vim e o compilador usado foi o g++.

    O grupo utilizou git para o desenvolvimento do trabalho, visto que
    precisávamos trabalhar de forma destrubuída, divindo tarefas para cada
    integrante do grupo, para que assim todos pudessem participar do trabalho.

    O repositório git encontra-se em: \href{https://github.com/LucianoPC/ping}{Repositório do projeto}

\section{Implementação}

    A implementação do cliente ping foi feita tanto em c++ quanto em python,
    para ambas as linguagens utizou-se da mesma lógica, primeiro foi feito uma
    função para calcular o RTT, que é o tempo que leva para enviar uma mensagem
    para o servidor e receber uma mensagem de volta, se a mensagem de volta não
    for recebida, ou seja, demorar mais de um segundo, é dado como falha na
    comunicação com o servidor, e a função do RTT retorna -1. Utilizando essa
    função que calcula um RTT foi implementado uma função que envia 10
    mensagens, uma após a outra, e mostra o RTT de cada mensagem, ou avisa que
    ocorreu uma falha na transmissão da mensagem entre o cliente e o servidor,
    devido ao RTT retornar o valor -1.

\section{Instruções de uso}

    Primeiramente deve-se executar o programa do servidor, para isso é
    necessário ter o python instalado no computador, e então no diretório raiz
    executar o comando './ping.py -s localhost', a porta padrão é a 2081,
    caso tenha algum problema com a porta, ela pode ser alterada, para isso
    deve-se executar o seguinte comando './ping -s localhost -p [server-port]'.

    Para executar o programa implementado em c++, primeiro deve-se compilar a
    aplicação, é necessário: possuir o compilador g++, estar no diretório raiz
    da aplicação e executar o comando make. Uma vez compilada, deve-se executar
    o comando './ping [server-address]', se necessário alterar a porta, execute
    o seguinte comando './ping [server-address] [server-port]'. Segue uma imagem
    com a execução do programa implementando em c++.

        \begin{figure}[h!]
            \centering
            \includegraphics[width=0.9\textwidth]{figuras/captura2.eps}
            \caption{Execução do programa implementado em c++}
        \end{figure}

    Para executar o programa implementado em python, basta executar o comando
    './ping.py [server-address]', se necessário alterar a porta, execute o
    seguinte comando './ping.py [server-address] -p [server-port]'. Segue uma
    imagem com a execução do programa implementando em python.

        \begin{figure}[h!]
            \centering
            \includegraphics[width=0.9\textwidth]{figuras/captura1.eps}
            \caption{Execução do programa implementado em python}
        \end{figure}
